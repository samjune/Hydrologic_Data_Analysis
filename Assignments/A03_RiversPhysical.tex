\documentclass[]{article}
\usepackage{lmodern}
\usepackage{amssymb,amsmath}
\usepackage{ifxetex,ifluatex}
\usepackage{fixltx2e} % provides \textsubscript
\ifnum 0\ifxetex 1\fi\ifluatex 1\fi=0 % if pdftex
  \usepackage[T1]{fontenc}
  \usepackage[utf8]{inputenc}
\else % if luatex or xelatex
  \ifxetex
    \usepackage{mathspec}
  \else
    \usepackage{fontspec}
  \fi
  \defaultfontfeatures{Ligatures=TeX,Scale=MatchLowercase}
\fi
% use upquote if available, for straight quotes in verbatim environments
\IfFileExists{upquote.sty}{\usepackage{upquote}}{}
% use microtype if available
\IfFileExists{microtype.sty}{%
\usepackage{microtype}
\UseMicrotypeSet[protrusion]{basicmath} % disable protrusion for tt fonts
}{}
\usepackage[margin=2.54cm]{geometry}
\usepackage{hyperref}
\hypersetup{unicode=true,
            pdftitle={Assignment 3: Physical Properties of Rivers},
            pdfauthor={Sam June},
            pdfborder={0 0 0},
            breaklinks=true}
\urlstyle{same}  % don't use monospace font for urls
\usepackage{color}
\usepackage{fancyvrb}
\newcommand{\VerbBar}{|}
\newcommand{\VERB}{\Verb[commandchars=\\\{\}]}
\DefineVerbatimEnvironment{Highlighting}{Verbatim}{commandchars=\\\{\}}
% Add ',fontsize=\small' for more characters per line
\usepackage{framed}
\definecolor{shadecolor}{RGB}{248,248,248}
\newenvironment{Shaded}{\begin{snugshade}}{\end{snugshade}}
\newcommand{\AlertTok}[1]{\textcolor[rgb]{0.94,0.16,0.16}{#1}}
\newcommand{\AnnotationTok}[1]{\textcolor[rgb]{0.56,0.35,0.01}{\textbf{\textit{#1}}}}
\newcommand{\AttributeTok}[1]{\textcolor[rgb]{0.77,0.63,0.00}{#1}}
\newcommand{\BaseNTok}[1]{\textcolor[rgb]{0.00,0.00,0.81}{#1}}
\newcommand{\BuiltInTok}[1]{#1}
\newcommand{\CharTok}[1]{\textcolor[rgb]{0.31,0.60,0.02}{#1}}
\newcommand{\CommentTok}[1]{\textcolor[rgb]{0.56,0.35,0.01}{\textit{#1}}}
\newcommand{\CommentVarTok}[1]{\textcolor[rgb]{0.56,0.35,0.01}{\textbf{\textit{#1}}}}
\newcommand{\ConstantTok}[1]{\textcolor[rgb]{0.00,0.00,0.00}{#1}}
\newcommand{\ControlFlowTok}[1]{\textcolor[rgb]{0.13,0.29,0.53}{\textbf{#1}}}
\newcommand{\DataTypeTok}[1]{\textcolor[rgb]{0.13,0.29,0.53}{#1}}
\newcommand{\DecValTok}[1]{\textcolor[rgb]{0.00,0.00,0.81}{#1}}
\newcommand{\DocumentationTok}[1]{\textcolor[rgb]{0.56,0.35,0.01}{\textbf{\textit{#1}}}}
\newcommand{\ErrorTok}[1]{\textcolor[rgb]{0.64,0.00,0.00}{\textbf{#1}}}
\newcommand{\ExtensionTok}[1]{#1}
\newcommand{\FloatTok}[1]{\textcolor[rgb]{0.00,0.00,0.81}{#1}}
\newcommand{\FunctionTok}[1]{\textcolor[rgb]{0.00,0.00,0.00}{#1}}
\newcommand{\ImportTok}[1]{#1}
\newcommand{\InformationTok}[1]{\textcolor[rgb]{0.56,0.35,0.01}{\textbf{\textit{#1}}}}
\newcommand{\KeywordTok}[1]{\textcolor[rgb]{0.13,0.29,0.53}{\textbf{#1}}}
\newcommand{\NormalTok}[1]{#1}
\newcommand{\OperatorTok}[1]{\textcolor[rgb]{0.81,0.36,0.00}{\textbf{#1}}}
\newcommand{\OtherTok}[1]{\textcolor[rgb]{0.56,0.35,0.01}{#1}}
\newcommand{\PreprocessorTok}[1]{\textcolor[rgb]{0.56,0.35,0.01}{\textit{#1}}}
\newcommand{\RegionMarkerTok}[1]{#1}
\newcommand{\SpecialCharTok}[1]{\textcolor[rgb]{0.00,0.00,0.00}{#1}}
\newcommand{\SpecialStringTok}[1]{\textcolor[rgb]{0.31,0.60,0.02}{#1}}
\newcommand{\StringTok}[1]{\textcolor[rgb]{0.31,0.60,0.02}{#1}}
\newcommand{\VariableTok}[1]{\textcolor[rgb]{0.00,0.00,0.00}{#1}}
\newcommand{\VerbatimStringTok}[1]{\textcolor[rgb]{0.31,0.60,0.02}{#1}}
\newcommand{\WarningTok}[1]{\textcolor[rgb]{0.56,0.35,0.01}{\textbf{\textit{#1}}}}
\usepackage{graphicx,grffile}
\makeatletter
\def\maxwidth{\ifdim\Gin@nat@width>\linewidth\linewidth\else\Gin@nat@width\fi}
\def\maxheight{\ifdim\Gin@nat@height>\textheight\textheight\else\Gin@nat@height\fi}
\makeatother
% Scale images if necessary, so that they will not overflow the page
% margins by default, and it is still possible to overwrite the defaults
% using explicit options in \includegraphics[width, height, ...]{}
\setkeys{Gin}{width=\maxwidth,height=\maxheight,keepaspectratio}
\IfFileExists{parskip.sty}{%
\usepackage{parskip}
}{% else
\setlength{\parindent}{0pt}
\setlength{\parskip}{6pt plus 2pt minus 1pt}
}
\setlength{\emergencystretch}{3em}  % prevent overfull lines
\providecommand{\tightlist}{%
  \setlength{\itemsep}{0pt}\setlength{\parskip}{0pt}}
\setcounter{secnumdepth}{0}
% Redefines (sub)paragraphs to behave more like sections
\ifx\paragraph\undefined\else
\let\oldparagraph\paragraph
\renewcommand{\paragraph}[1]{\oldparagraph{#1}\mbox{}}
\fi
\ifx\subparagraph\undefined\else
\let\oldsubparagraph\subparagraph
\renewcommand{\subparagraph}[1]{\oldsubparagraph{#1}\mbox{}}
\fi

%%% Use protect on footnotes to avoid problems with footnotes in titles
\let\rmarkdownfootnote\footnote%
\def\footnote{\protect\rmarkdownfootnote}

%%% Change title format to be more compact
\usepackage{titling}

% Create subtitle command for use in maketitle
\providecommand{\subtitle}[1]{
  \posttitle{
    \begin{center}\large#1\end{center}
    }
}

\setlength{\droptitle}{-2em}

  \title{Assignment 3: Physical Properties of Rivers}
    \pretitle{\vspace{\droptitle}\centering\huge}
  \posttitle{\par}
    \author{Sam June}
    \preauthor{\centering\large\emph}
  \postauthor{\par}
    \date{}
    \predate{}\postdate{}
  

\begin{document}
\maketitle

\hypertarget{overview}{%
\subsection{OVERVIEW}\label{overview}}

This exercise accompanies the lessons in Hydrologic Data Analysis on the
physical properties of rivers.

\hypertarget{directions}{%
\subsection{Directions}\label{directions}}

\begin{enumerate}
\def\labelenumi{\arabic{enumi}.}
\tightlist
\item
  Change ``Student Name'' on line 3 (above) with your name.
\item
  Work through the steps, \textbf{creating code and output} that fulfill
  each instruction.
\item
  Be sure to \textbf{answer the questions} in this assignment document.
\item
  When you have completed the assignment, \textbf{Knit} the text and
  code into a single PDF file.
\item
  After Knitting, submit the completed exercise (PDF file) to the
  dropbox in Sakai. Add your last name into the file name (e.g.,
  ``Salk\_A03\_RiversPhysical.Rmd'') prior to submission.
\end{enumerate}

The completed exercise is due on 18 September 2019 at 9:00 am.

\hypertarget{setup}{%
\subsection{Setup}\label{setup}}

\begin{enumerate}
\def\labelenumi{\arabic{enumi}.}
\tightlist
\item
  Verify your working directory is set to the R project file,
\item
  Load the tidyverse, dataRetrieval, and cowplot packages
\item
  Set your ggplot theme (can be theme\_classic or something else)
\item
  Import a data frame called ``MysterySiteDischarge'' from USGS gage
  site 03431700. Upload all discharge data for the entire period of
  record. Rename columns 4 and 5 as ``Discharge'' and ``Approval.Code''.
  DO NOT LOOK UP WHERE THIS SITE IS LOCATED.
\item
  Build a ggplot of discharge over the entire period of record.
\end{enumerate}

\begin{Shaded}
\begin{Highlighting}[]
\KeywordTok{getwd}\NormalTok{()}
\end{Highlighting}
\end{Shaded}

\begin{verbatim}
## [1] "/Users/samjune/Desktop/Fall 2019/Hydrologic Data Analysis/Hydrologic_Data_Analysis/Assignments"
\end{verbatim}

\begin{Shaded}
\begin{Highlighting}[]
\KeywordTok{library}\NormalTok{(tidyverse)}
\end{Highlighting}
\end{Shaded}

\begin{verbatim}
## -- Attaching packages ------------------------------------------------------ tidyverse 1.2.1 --
\end{verbatim}

\begin{verbatim}
## v ggplot2 3.2.1     v purrr   0.3.2
## v tibble  2.1.3     v dplyr   0.8.3
## v tidyr   0.8.3     v stringr 1.4.0
## v readr   1.3.1     v forcats 0.4.0
\end{verbatim}

\begin{verbatim}
## -- Conflicts --------------------------------------------------------- tidyverse_conflicts() --
## x dplyr::filter() masks stats::filter()
## x dplyr::lag()    masks stats::lag()
\end{verbatim}

\begin{Shaded}
\begin{Highlighting}[]
\KeywordTok{library}\NormalTok{(dataRetrieval)}
\KeywordTok{library}\NormalTok{(cowplot)}
\end{Highlighting}
\end{Shaded}

\begin{verbatim}
## 
## ********************************************************
\end{verbatim}

\begin{verbatim}
## Note: As of version 1.0.0, cowplot does not change the
\end{verbatim}

\begin{verbatim}
##   default ggplot2 theme anymore. To recover the previous
\end{verbatim}

\begin{verbatim}
##   behavior, execute:
##   theme_set(theme_cowplot())
\end{verbatim}

\begin{verbatim}
## ********************************************************
\end{verbatim}

\begin{Shaded}
\begin{Highlighting}[]
\KeywordTok{theme_set}\NormalTok{(}\KeywordTok{theme_cowplot}\NormalTok{())}

\NormalTok{MysterySiteDischarge <-}\StringTok{ }\KeywordTok{readNWISdv}\NormalTok{(}\DataTypeTok{siteNumbers =} \StringTok{"03431700"}\NormalTok{,}
                        \DataTypeTok{parameterCd =} \StringTok{"00060"}\NormalTok{, }\CommentTok{# discharge (ft3/s)}
                        \DataTypeTok{startDate =} \StringTok{""}\NormalTok{,}
                        \DataTypeTok{endDate =} \StringTok{""}\NormalTok{)}

\KeywordTok{colnames}\NormalTok{(MysterySiteDischarge)[}\KeywordTok{colnames}\NormalTok{(MysterySiteDischarge)}\OperatorTok{==}
\StringTok{                                 "X_00060_00003"}\NormalTok{]<-}\StringTok{"Discharge"}
\KeywordTok{colnames}\NormalTok{(MysterySiteDischarge)[}\KeywordTok{colnames}\NormalTok{(MysterySiteDischarge)}\OperatorTok{==}
\StringTok{                                 "X_00060_00003_cd"}\NormalTok{]<-}\StringTok{"Approval.Code"}

\NormalTok{MysterySitePlot <-}\StringTok{ }
\StringTok{  }\KeywordTok{ggplot}\NormalTok{(MysterySiteDischarge, }\KeywordTok{aes}\NormalTok{(}\DataTypeTok{x =}\NormalTok{ Date, }\DataTypeTok{y =}\NormalTok{ Discharge)) }\OperatorTok{+}
\StringTok{         }\KeywordTok{geom_line}\NormalTok{() }\OperatorTok{+}
\StringTok{         }\KeywordTok{xlab}\NormalTok{(}\StringTok{"Year"}\NormalTok{)}
\KeywordTok{print}\NormalTok{(MysterySitePlot)}
\end{Highlighting}
\end{Shaded}

\includegraphics{A03_RiversPhysical_files/figure-latex/unnamed-chunk-1-1.pdf}

\hypertarget{analyze-seasonal-patterns-in-discharge}{%
\subsection{Analyze seasonal patterns in
discharge}\label{analyze-seasonal-patterns-in-discharge}}

\begin{enumerate}
\def\labelenumi{\arabic{enumi}.}
\setcounter{enumi}{4}
\tightlist
\item
  Add a ``Year'' and ``Day.of.Year'' column to the data frame.
\item
  Create a new data frame called ``MysterySiteDischarge.Pattern'' that
  has columns for Day.of.Year, median discharge for a given day of year,
  75th percentile discharge for a given day of year, and 25th percentile
  discharge for a given day of year. Hint: the summarise function
  includes \texttt{quantile}, wherein you must specify \texttt{probs} as
  a value between 0 and 1.
\item
  Create a plot of median, 75th quantile, and 25th quantile discharges
  against day of year. Median should be black, other lines should be
  gray.
\end{enumerate}

\begin{Shaded}
\begin{Highlighting}[]
\KeywordTok{library}\NormalTok{(lubridate)}
\end{Highlighting}
\end{Shaded}

\begin{verbatim}
## 
## Attaching package: 'lubridate'
\end{verbatim}

\begin{verbatim}
## The following object is masked from 'package:cowplot':
## 
##     stamp
\end{verbatim}

\begin{verbatim}
## The following object is masked from 'package:base':
## 
##     date
\end{verbatim}

\begin{Shaded}
\begin{Highlighting}[]
\NormalTok{MysterySiteDischarge <-}\StringTok{ }
\StringTok{  }\NormalTok{MysterySiteDischarge }\OperatorTok
\StringTok{  }\KeywordTok{mutate}\NormalTok{(}\DataTypeTok{Year =} \KeywordTok{year}\NormalTok{(Date))}

\NormalTok{MysterySiteDischarge <-}\StringTok{ }
\StringTok{  }\NormalTok{MysterySiteDischarge }\OperatorTok
\StringTok{  }\KeywordTok{mutate}\NormalTok{(}\DataTypeTok{Day.of.Year =} \KeywordTok{yday}\NormalTok{(Date))}

\NormalTok{MysterySiteDischarge.Pattern <-}\StringTok{ }\NormalTok{MysterySiteDischarge[,}\KeywordTok{c}\NormalTok{(}\DecValTok{7}\NormalTok{)]}
\end{Highlighting}
\end{Shaded}

\begin{enumerate}
\def\labelenumi{\arabic{enumi}.}
\setcounter{enumi}{7}
\tightlist
\item
  What seasonal patterns do you see? What does this tell you about
  precipitation patterns and climate in the watershed?
\end{enumerate}

\begin{quote}
\end{quote}

\hypertarget{create-and-analyze-recurrence-intervals}{%
\subsection{Create and analyze recurrence
intervals}\label{create-and-analyze-recurrence-intervals}}

\begin{enumerate}
\def\labelenumi{\arabic{enumi}.}
\setcounter{enumi}{8}
\item
  Create two separate data frames for MysterySite.Annual.30yr (first 30
  years of record) and MysterySite.Annual.Full (all years of record).
  Use a pipe to create your new data frame(s) that includes the year,
  the peak discharge observed in that year, a ranking of peak
  discharges, the recurrence interval, and the exceedende probability.
\item
  Create a plot that displays the discharge vs.~recurrence interval
  relationship for the two separate data frames (one set of points
  includes the values computed from the first 30 years of the record and
  the other set of points includes the values computed for all years of
  the record.
\item
  Create a model to predict the discharge for a 100-year flood for both
  sets of recurrence intervals.
\item
  How did the recurrence interval plots and predictions of a 100-year
  flood differ among the two data frames? What does this tell you about
  the stationarity of discharge in this river?
\end{enumerate}

\begin{quote}
\end{quote}

\hypertarget{reflection}{%
\subsection{Reflection}\label{reflection}}

\begin{enumerate}
\def\labelenumi{\arabic{enumi}.}
\setcounter{enumi}{12}
\tightlist
\item
  What are 2-3 conclusions or summary points about river discharge you
  learned through your analysis?
\end{enumerate}

\begin{quote}
\end{quote}

\begin{enumerate}
\def\labelenumi{\arabic{enumi}.}
\setcounter{enumi}{13}
\tightlist
\item
  What data, visualizations, and/or models supported your conclusions
  from 13?
\end{enumerate}

\begin{quote}
\end{quote}

\begin{enumerate}
\def\labelenumi{\arabic{enumi}.}
\setcounter{enumi}{14}
\tightlist
\item
  Did hands-on data analysis impact your learning about discharge
  relative to a theory-based lesson? If so, how?
\end{enumerate}

\begin{quote}
\end{quote}

\begin{enumerate}
\def\labelenumi{\arabic{enumi}.}
\setcounter{enumi}{15}
\tightlist
\item
  How did the real-world data compare with your expectations from
  theory?
\end{enumerate}

\begin{quote}
\end{quote}


\end{document}
